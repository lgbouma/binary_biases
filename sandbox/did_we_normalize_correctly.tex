\documentclass[12pt,modern]{aastex61}
\usepackage{graphics,graphicx}
\usepackage{hyperref}
\usepackage{amssymb}
\usepackage{amsmath}
\usepackage{comment}

\begin{document}
    
\newcommand{\pt}{\theta}
\newcommand{\pa}{\theta_{\rm a}}
\newcommand{\pn}{\theta_0}

\newcommand{\pp}{\mathcal{P}}
\newcommand{\ps}{\mathcal{S}}
\renewcommand{\a}{_{\rm a}}
\newcommand{\s}{_{\rm s}}

\section{Did we normalize the number of searchable stars correctly?}

Recall how we evaluated
\begin{align}
N_{\rm det}(r\a, M\a) &=
\sum_i N_{\rm det}^i(r\a, M\a) \\
N_{\rm det}(r\a, M\a)
&=
\sum_i \int \mathrm{d}r \mathrm{d}M \,
\mathcal{N}_{\rm s}^i(r\a,M\a; r,M)
\cdot\Gamma^i(r,M) \cdot p_{\rm tra}(r,M).
\end{align}
The fancy-looking $\mathcal{N}_{\rm s}^i(r\a,M\a; r,M)$ term is supposed to 
mean ``the number of searchable stars of type $i$ (per unit $r\a,M\a$) that 
give $(r\a,M\a)$ when the true system actually has $(r,M)$''.

For $i=0$, we wrote
\begin{equation}
\mathcal{N}_{\rm s}^0(r\a,M\a; r,M)
=\delta(r\a-r)\delta(M\a-M) N\s^0(r,M),
\label{eq:simple}
\end{equation}
where $N\s^0(r,M)$ is the number of searchable singles with true parameters 
$(r,M)$.
Note that this $N\s^0(r,M)$ should indeed be written a function of the {\it 
true} parameters, and only once integrated will it give a function of apparent 
parameters.

We then get
\begin{align}
\notag
N_{\rm det}^0(r\a,M\a)
&= \int {\rm d}r{\rm d}M 
	\mathcal{N}_{\rm s}^0(r\a,M\a; r,M) \Gamma^0(r,M) p_{\rm tra}(r,M) \\
\notag
&= \int {\rm d}r{\rm d}M 
\delta(r\a-r)\delta(M\a-M) N\s^0(r,M) \Gamma^0(r,M) p_{\rm tra}(r,M) \\
&= N\s^0(r\a,M\a) \Gamma^0(r\a,M\a) p_{\rm tra}(M\a).
\end{align}

For $i=1$, by definition,
\begin{equation}
N_{\rm det}^1(r\a,M\a) =
\int \mathrm{d}r \mathrm{d}M \,
\mathcal{N}_{\rm s}^1(r\a,M\a; r,M)
\cdot\Gamma^i(r,M) \cdot p_{\rm tra}(r,M).
\end{equation}
Just as before, we must marginalize out the binary distribution:
\begin{equation}
\mathcal{N}_{\rm s}^1(r_a, M_a; r, M)
=\int \mathrm{d}q\,f(q)\mathcal{N}_{{\rm s}, q}^1(r_a, M_a; r, M; q).
\end{equation}
We can thus write (order of integration does not matter)
\begin{equation}
N_{\rm det}^1(r\a,M\a) =
\int \mathrm{d}r \mathrm{d}M \mathrm{d}q \,
f(q)
\mathcal{N}_{\rm s}^1(r\a,M\a; r,M; q)
\cdot\Gamma^i(r,M) \cdot p_{\rm tra}(r,M).
\label{eq:N_det_1}
\end{equation}
Now I'm pretty sure that we need to write
\begin{equation}
\mathcal{N}_{\rm s}^1(r\a,M\a; r,M; q)
=\delta(r\a-r\mathcal{A}(q))\delta(M\a-M) N\s^1(r,M;q),
\label{eq:less_simple}
\end{equation}
where $\mathcal{A}(q)$ is the necessary correction for dilution, defined 
identically as before, but now $N\s^1(r,M;q)$ is the number of primaries that 
are searchable for planets when the system has true parameters $(r,M,q)$.

This is immediately different from what we've done previously.
We previously wrote
\begin{equation}
	N\s^i(\delta_{\rm obs}, L_{\rm sys}) \propto n_i \delta_{\rm obs}^3 L_{\rm 
		sys}^{3/2},
\end{equation}
where $N\s^i(\delta_{\rm obs}, L_{\rm sys})$ is the number of searchable 
stars of type $i$, given {\it apparent} parameters.

Now, instead, when I write $N\s^1(r,M;q)$, I'm saying: {\it given true 
parameters}, how many searchable primaries are there? I'm pretty sure this 
latter question is the correct one to ask for purposes of determining 
$\mathcal{N}_{\rm s}^1(r\a,M\a; r,M; q)$.
{\bf If you disagree, then don't bother reading the rest}.
Assuming we're asking the right question, let's find out the effect on the 
$\mathcal{N}\s^1$ quantity. First, note
\begin{equation}
d_{\rm det}^1(r,R,M,q) \propto
\left(\frac{L(M\a)}{L(M\a)+L(qM\a)}\right)
\cdot \left(\frac{r}{R}\right)^2
\cdot \left(L(M\a)+L(qM\a)\right)^{1/2},
\end{equation}
where I ignored the $T_{\rm dur}$ dependence, and wrote the terms for 
dilution, geometric transit depth, and system luminosity.
Taking the cube,
\begin{equation}
N\s^1(r,R,M,q) \propto
n_b
\frac{L(M\a)^3}{(L(M\a) + L(qM\a))^{3/2}}
\left(\frac{r}{R}\right)^6.
\label{eq:N_s_real_parameters}
\end{equation}
This number of searchable stars, given the true system parameters, will be 
what goes into determining the number of detections (given apparent 
parameters).

I think this also means $f(q)$ in Eq.~\ref{eq:N_det_1} should be $f_{\rm 
vl}(q)$~---~the volume limited binary distribution.
Before, we (I) said it would be magnitude limited.
The reasoning for why it must be volume-limited will become clear in 
Eq.~\ref{eq:mag_limit_becomes_clear}, where it will be evident that doing 
otherwise would be erroneously accounting for the Malmquist bias twice.

Let's compute the contribution of primaries to the overall rate density,
\begin{equation}
\Gamma\a^1(r\a,M\a) = \frac{N_{\rm det}^1(r\a,M\a)}{N_{\rm s,a}(r\a,M\a)} 
\frac{1}{p_{\rm tra}(r\a,M\a)}.
\label{eq:Gamma_a_1}
\end{equation}
As before, we have
\begin{equation}
N_{\rm s,a}(r\a,M\a) = N\s^0(r\a,M\a)(1+\mu),
\end{equation}
where $\mu \equiv N_d/N_s^0$ is, as we've agreed before, a number that 
is determined by the binary mass ratio distribution, the binary fraction, and 
nothing else.

From Eq.~\ref{eq:N_det_1} and Eq.~\ref{eq:less_simple}, we now get
\begin{equation}
N_{\rm det}^1(r\a,M\a) =
p_{\rm tra}(M\a)
\int \frac{{\rm d}q\,f_{\rm vl}(q)}{\mathcal{A}(q)}
N\s^1\left(r=\frac{r\a}{\mathcal{A}(q)},M=M\a, q=q\right)
\Gamma^1\left(\frac{r\a}{\mathcal{A}(q)},M\a\right),
\end{equation}
where I'm keeping the explicit $N\s^1(r,M,q)$ dependence to indicate I am 
referring to Eq.~\ref{eq:N_s_real_parameters}, the number of searchable 
primaries given real parameters.
Plugging into Eq.~\ref{eq:Gamma_a_1},
\begin{align}
\Gamma\a^1(r\a,M\a) &=
\frac{1}{1+\mu}
\int \frac{{\rm d}q\,f_{\rm vl}(q)}{\mathcal{A}(q)}
\frac{N\s^1\left(r=\frac{r\a}{\mathcal{A}(q)},M=M\a, q=q\right)}
{N\s^0(r\a,M\a)}
\Gamma^1\left(\frac{r\a}{\mathcal{A}(q)},M\a\right) \\
&=
\frac{(n_b/n_s)}{1+\mu}
\int \frac{{\rm d}q\,f_{\rm vl}(q)}{\mathcal{A}(q)}
\left(\frac{L(M\a)}{L(M\a)+L(qM\a)}\right)^{3/2}
\mathcal{A}(q)^{-6}
\Gamma^1\left(\frac{r\a}{\mathcal{A}(q)},M\a\right) 
\label{eq:mag_limit_becomes_clear}\\
&=
\frac{(n_b/n_s)}{1+\mu}
\int {\rm d}q\,f_{\rm vl}(q)
\mathcal{A}(q)^{-4}
\Gamma^1\left(\frac{r\a}{\mathcal{A}(q)},M\a\right).
\end{align}

If you compute out the various limiting cases, it turns out for $f_{\rm 
vl}(q)=\delta(q-1)$, the above is exactly the same as what we had before.
However, if $f_{\rm vl}(q)=q^\beta / \mathcal{N}_q$, it is slightly different.


\end{document}
