\documentclass[12pt,modern]{aastex61}
\usepackage{graphics,graphicx}
\usepackage{hyperref}
\usepackage{amssymb}
\usepackage{amsmath}
\usepackage{comment}

%% Reintroduced the \received and \accepted commands from AASTeX v5.2
%\received{July 1, 2016}
%\revised{September 27, 2016}
%\accepted{\today}
%% Command to document which AAS Journal the manuscript was submitted to.
%% Adds "Submitted to " the arguement.
\submitjournal{AAS journals.}


\shortauthors{Bouma et al.}
\shorttitle{Binarity and Occurrence Rates}

\begin{document}
    
\title{ The effects of binarity on planet occurrence rates measured by transit 
surveys}
%
\correspondingauthor{L. Bouma}
\email{luke@astro.princeton.edu}
%
\author{L. G. Bouma}
\affiliation{
    Department of Astrophysical Sciences,
    Princeton University,
    4 Ivy Lane, Princeton, NJ 08540, USA}
\author{J. N. Winn}
\affiliation{
    Department of Astrophysical Sciences,
    Princeton University,
    4 Ivy Lane, Princeton, NJ 08540, USA}
\author{K. Masuda}
\affiliation{
    Department of Astrophysical Sciences,
    Princeton University,
    4 Ivy Lane, Princeton, NJ 08540, USA}
%
%
\begin{abstract}
%
This derives the equations of the paper.

%
\end{abstract}
%
\keywords{
    methods: data analysis ---
    planets and satellites: detection ---
    surveys}
%
%

\newcommand{\pt}{\theta}
\newcommand{\pa}{\theta_{\rm a}}
\newcommand{\pn}{\theta_0}

\newcommand{\pp}{\mathcal{P}}
\newcommand{\ps}{\mathcal{S}}
\renewcommand{\a}{_{\rm a}}
\newcommand{\s}{_{\rm s}}

\section{Preliminaries}

\subsection{Searchable Distance}

We can detect a signal if 
\begin{equation}
	{\text{signal}\over\text{noise}}\sim
    {\delta_{\rm obs}\over(L/d^2)^{-1/2}},
	\quad \delta_{\rm obs}:\text{observed depth}
\end{equation}
is above some threshold (it would probably make more sense to include the 
duration information, but that would anyway be a trivial extension and so we 
omit it here for brevity). Thus, the maximum searchable distance scales as
\begin{equation}
	d(\delta_{\rm obs}, L_{\rm sys}) \propto \delta_{\rm obs} \cdot L_{\rm 
	sys}^{1/2}.
\end{equation}
We assume that the signal is detected if and only if a given star is searchable. 

Using this distance, the number of searchable stars $N\s$ is given by
\begin{equation}
	N\s(\delta_{\rm obs}, L_{\rm sys})=n_0 \delta_{\rm obs}^3 L_{\rm 
	sys}^{3/2},
\end{equation}
where $n_0$ is a normalization constant proportional to the volume density of the systems with a certain type of interest (e.g. single star, binary).
We neglect the dependence of the normalization $n_0$ on the stellar type.

\subsection{Relation between Apparent and Actual Stellar Properties}

We assume that the apparent property of an unresolved binary is the same as that of the primary:
\begin{equation}
	M\a=M_1, \quad R\a=R_1, ...
\end{equation}
We also assume the stellar radius and luminosity is uniquely related to the stellar mass.

Given these assumptions, the total luminosity of the system is
%This assumption, along with a given $L$--$M$ relation, leads to
\begin{equation}
	L_{\rm sys}=L_1+L_2=L(M\a)+L(qM\a),
\end{equation}
where $q=M_2/M_1$. Note that $L_{\rm sys}$ is the true value (because $M\a$ is 
the true primary mass), while the system luminosity estimated by an observer 
would be $L(M\a)$, which based on the apparent stellar parameters (unless the 
observer has a priori knowledge on the distance to a given system).

\subsection{Apparent Number of Searchable Stars}

Given the apparent signal $\delta_{\rm obs}$ and stellar mass $M\a$, the 
maximum searchable distance for singles and binaries are proportional to 
$\delta_{\rm obs}\cdot L(M\a)^{1/2}$ and $\delta_{\rm obs}\cdot 
[L(M\a)+L(qM\a)]^{1/2}$. Thus, 
the apparent number of searchable stars (i.e. points in the sky), which will 
be selected by an ignorant observer, is 
\begin{align}
	\notag
	N_{\rm s,a}(\delta_{\rm obs}, M\a)
	%&=n_{\rm s,0}\delta_{\rm obs}^3L(M\a)^{3/2} + n_{\rm b,0}\delta_{\rm 
	%obs}^3 \cdot [L(M\a)+L(qM\a)]^{3/2}\\
	&=n_{\rm s}\delta_{\rm obs}^3 L(M\a)^{3/2}\left\{1+\int \mathrm{d}q\,f(q) 
	{\mathrm{BF}\over{1-\mathrm{BF}}}\left[1+{L(qM\a) \over 
	L(M\a)}\right]^{3/2}\right\}\\
	&\equiv N\s^0(\delta_{\rm obs}, L(M\a)) \left[1+\mu(\mathrm{BF}, 
	M\a)\right],
\end{align}
where $N\s^0$ is the number of searchable singles (this agrees with the actual value), $f(q)$ is the binary mass ratio distribution, and $\mathrm{BF}=n_{\rm b}/(n_{\rm s}+n_{\rm b})$ is the binary fraction in a volume-limited sample.

\section{Apparent Occurrence Rate --- General Formula}

A group of astronomers wants to measure the mean number of planets of a 
certain type per star of a certain type.
They simply observe a set points on the sky selected in a magnitude-limited way (or whatever, actually?) and detect $n$ planets of a desired class. Then they compute the occurrence as
\begin{equation}
\Lambda(\pp, \ps) = \frac{n(\pp, \ps)}{N_{\rm s}(\pp, \ps)} \times \frac{1}{p_{\rm tra}(\pp, \ps)}.
\end{equation}
Here $N_{\rm s}$ is the number of stars (among those initially selected) around which the planets of interest are searchable; $p_{\rm tra}$ is the transit probability.

Let's see how the binary modifies this. The occurrence is computed based on the apparent planetary/stellar parameters $\pp\a$, $\ps\a$:
\begin{equation}
\Lambda\a(\pp\a, \ps\a) = \frac{n(\pp\a, \ps\a)}{N_{\rm s,a}(\pp\a, \ps\a)} \times \frac{1}{p_{\rm tra}(\pp\a, \ps\a)}.
\end{equation}
Here, $N_{\rm s}$ and $p_{\rm tra}$ are computed for the apparent parameter $\pp\a$ and $\ps\a$. In the presence of dilution, planets with $(\pp\a, \ps\a)$ are associated with systems of many different planetary and stellar properties, so $n_{\rm a}$ is given by the convolution of the true occurrence, $\Lambda(\pp, \ps)$, and number of searchable stars that give $(\pp\a, \ps\a)$ when the true system actually has $(\pp, \ps)$, $\mathcal{N}(\pp\a, \ps\a; \pp, \ps)$:
\begin{equation}
	n(\pp\a, \ps\a)=\sum_i n^i(\pp\a, \ps\a)
	=\sum_i \int \mathrm{d}\pp\mathrm{d}\ps\,
	%\mathcal{P}_{\rm tra}(\pp\a, \ps\a; \pt)
	\mathcal{N}_{\rm s}^i(\pp\a,\ps\a; \pp, \ps)
	\cdot\Lambda^i(\pp, \ps)\cdot p_{\rm tra}(\pp, \ps),
\end{equation}
where $i$ specifies the type of true host stars (0: single, 1: primary, 2: secondary).
%where $b_i=1-\mathrm{BF}$ for $i=0$ (singles) and $b_i=\mathrm{BF}$ for $i=1,2$. In fact, the $b_i$ factor appears when we count the number of searchable systems $\mathcal{N}$; what matters here is the number density of binaries and singles, so $\mathrm{BF}$ is the binary fraction in a volume-limited sample. Here we factor it out just to simplify(?) the notation.

So the problem essentially reduces to the evaluation of
\begin{equation}
	\mathcal{N}_{\rm s}(\pp\a,\ps\a; \pp, \ps)
\end{equation}
for planets around single stars, primaries in binaries, and secondaries in binaries. 

\section{Evaluation of $\mathcal{N}$}

Let us explicitly write $\pp=r$ and $\ps=M$; $R$ and $L$ are uniquely determined from the assumed mass--radius--luminosity relation. We neglect the $P$ dependence.

\subsection{Single Stars}

For $i=0$, 
\begin{equation}
	\mathcal{N}_{\rm s}^0(\pp\a,\ps\a; \pp, \ps)
	=\delta(\pp\a-\pp)\delta(\ps\a-\ps) N\s^0(\pp,\ps),
	%=\delta(r'-r)\delta(M'-M)\delta(R'-R) N\s((r/R)^2, L(M))
\end{equation}
so
\begin{equation}
	n^0(\pp\a, \ps\a)=N\s^0(\pp\a,\ps\a)\cdot \Lambda^0(\pp\a, \ps\a) \cdot p_{\rm tra}(\pp\a, \ps\a).
\end{equation}
If all the stars are singles, this yields
\begin{equation}
	\Lambda\a(\pp\a, \ps\a)=\Lambda^0(\pp\a, \ps\a),
\end{equation}
as expected (now $\mu=0$ and $N_{\rm s,a}=N\s^0$) --- the true occurrence is recovered.

\subsection{Primaries in Binaries}

Since we assume $\ps\a=\ps_1$,
\begin{equation}
	\mathcal{N}_{\rm s}^1(r', M'; r, M) \propto \delta(M'-M).
	%=\mathcal{N}_{\rm s}^1(r', M', R'; r, M', R').
\end{equation}
In this case, $\mathcal{N}_{\rm s}^1$ is non-zero only at $r'=R'\sqrt{D'}$, where
\begin{equation}
	D' %= \delta \times {L_1 \over L_{\rm sys}} 
	= \left[{r\over R(M')}\right]^2\times {L(M') \over L_{\rm sys}(M', q)}
\end{equation}
and the normalization is given by the number of searchable binaries (for signal $D'$):
\begin{equation}
	%N_{\rm s}(D', L_{\rm sys})
	n_{\rm b}D'^3 L_{\rm sys}^{3/2}=N_{\rm s}^0(D', L(M'))\cdot\mu(\mathrm{BF},M').
	%=N_{\rm s}(D', L(M')) \times \left[L_{\rm sys}(M', q) \over L(M') 
	%\right]^{3/2}.
\end{equation}
Thus,
\begin{equation}
	\mathcal{N}_{\rm s}^1(r', M'; r, M)
	=\int \mathrm{d}q\,f(q)\mathcal{N}_{{\rm s}, q}^1(r', M'; r, M; q),
\end{equation}
where $f(q)$ is the binary mass ratio distribution and
\begin{align}
	\notag
	\mathcal{N}_{{\rm s}, q}^1(r', M'; r, M; q)
	%&=N_{\rm s}(D', L(M')) \left[L_{\rm sys}(M', q) \over L(M') 
	%\right]^{3/2}\\
	&=N_{\rm s}^0(D', L(M'))\cdot\mu(\mathrm{BF},M')\\
	&\times\delta \left(r'-r\sqrt{{L(M') \over L_{\rm sys}(M', 
	q)}}\right)\delta(M'-M).
\end{align}

\subsection{Secondaries in Binaries}

In this case, $M=qM_1=qM'$, so
\begin{equation}
	\mathcal{N}_{\rm s}^2(r', M'; r, M)
	\propto \delta\left(M'-{M\over q}\right).
	%=\mathcal{N}_{\rm s}^2(r', M', R'; r, qM', R(qM')).
\end{equation}
Again $\mathcal{N}_{\rm s}^2$ is non-zero only at $r'=R'\sqrt{D'}$, but this time
\begin{equation}
	D' = \left[{r\over R(qM')}\right]^2 \times {L(qM') 
	\over L_{\rm sys}(M', q)}.
\end{equation}
The normalization remains the same as the previous case (we are counting the searchable stars at a given observed depth $D'$, total luminosity of the binary is the same).
Thus,
\begin{equation}
	\mathcal{N}_{\rm s}^2(r', M'; r, M)
	=\int \mathrm{d}q\,f(q)\mathcal{N}_{{\rm s}, q}^2(r', M'; r, M; q),
\end{equation}
where
\begin{align}
	\notag
	\mathcal{N}_{\rm s}^2(r', M'; r, M; q)
	%&=N_{\rm s}(D', L(M')) \left[L_{\rm sys}(M', q) \over L(M') 
	%\right]^{3/2}\\
	&=N_{\rm s}^0(D', L(M'))\cdot\mu(\mathrm{BF},M')\\
	&\times \delta \left(r'-r\sqrt{ \left[{R(M')\over R(qM')}\right]^2 {L(qM') 
	\over L_{\rm sys}(M', q)} }\right)\delta\left(M'-{M\over q}\right).
\end{align}

\section{Result}

\subsection{Marginalization over the True Properties}

Let's integrate out $\pp$ and $\ps$.

\begin{align}
	n^0(r\a, M\a)
	&=\int\mathrm{d}r\mathrm{d}M\,\mathcal{N}_{\rm s}^0(r\a, M\a; r, M)
	\cdot\Lambda^0(r, M) \cdot p_{\rm tra}(M)\\
	&=\int\mathrm{d}r\mathrm{d}M\, N\s^0(\delta_{\rm obs}, 
	L(M\a))\delta(r\a-r)\delta(M\a-M)
	\cdot\Lambda^0(r, M) \cdot p_{\rm tra}(M)\\
	&=N\s^0(\delta_{\rm obs}, L(M\a))\cdot\Lambda^0(r\a, M\a) \cdot p_{\rm 
	tra}(M\a).
\end{align}

\begin{align}
	n^1(r\a, M\a)
	&=\int\mathrm{d}r\mathrm{d}M\,\mathcal{N}_{\rm s}^1(r\a, M\a; r, M)
	\cdot\Lambda^1(r, M) \cdot p_{\rm tra}(M)\\
	&=\int \mathrm{d}q\,f(q)\int\mathrm{d}r\mathrm{d}M\,\mathcal{N}_{{\rm s}, q}^1(r\a, M\a; r, M; q)\cdot\Lambda^1(r, M) \cdot p_{\rm tra}(M)\\
	&=N\s^0(\delta_{\rm obs}, L(M\a))\cdot p_{\rm tra}(M\a) \cdot
	\mu(\mathrm{BF}, M\a) \int {\mathrm{d}q \over \alpha}\,f(q)\,\Lambda^1(r^*, M\a),
	%\left[L_{\rm sys}(M', q) \over L(M') \right]^{3/2},
\end{align}
where $r^*=r\a/\alpha$ and 
\begin{equation}
	\alpha(q, M\a)=\sqrt{L(M\a) \over L_{\rm sys}(M\a, q)}.
\end{equation}

Finally,
\begin{align}
	n^2(r\a, M\a)
	&=\int\mathrm{d}r\mathrm{d}M\,\mathcal{N}_{\rm s}^2(r\a, M\a; r, M)
	\cdot\Lambda^2(r, M) \cdot p_{\rm tra}(M)\\
	&=\int \mathrm{d}q\,f(q)\int\mathrm{d}r\mathrm{d}M\,\mathcal{N}_{{\rm s}, q}^2(r\a, M\a; r, M; q)\cdot\Lambda^2(r, M) \cdot p_{\rm tra}(M)\\
	&=%N_{\rm s}(D', L(M')
	N\s^0(\delta_{\rm obs}, L(M\a))\cdot\mu(\mathrm{BF}, M\a)
	\int {q \mathrm{d}q \over \beta}\,f(q)\,\Lambda^2(r^{**}, qM\a)\,p_{\rm tra}(qM\a),
	%\left[L_{\rm sys}(M', q) \over L(M') \right]^{3/2},
\end{align}
where $r^{**}=r\a/\beta$ and 
\begin{equation}
	\beta(q, M\a)={R(M\a)\over R(qM\a)}\sqrt{{L(qM\a) \over L_{\rm sys}(M\a, q)} }.
\end{equation}

\subsection{Final Formula}

Note again that the denominators of $\Lambda\a$ are the same for singles, 
primaries, and secondaries: $N_{\rm s,a}(\delta_{\rm obs}, M\a)$ and $p_{\rm 
tra}(M\a)$. This is because we are calculating the occurrence at the same 
apparent planet/star properties, and the observer can never distinguish 
binaries from singles (and adopt the primary properties for binaries).
Using the results above, the apparent occurrence is thus given by
\begin{align}
	&\Lambda\a(r\a, M\a)={1\over 1+\mu(\mathrm{BF}, M\a)}\times\\
	%&=(1-\mathrm{BF})\cdot
	&\left\{ \Lambda^0(r\a, M\a)+ \mu(\mathrm{BF}, M\a) \left[
	\int {\mathrm{d}q \over A}\,f(q)\,\Lambda^1\left({r\a\over A}, M\a\right)\,
	+\int {q \mathrm{d}q \over B}\,f(q)\,\Lambda^2\left({r\a\over B}, qM\a\right)\,
	{R(qM\a) \over R(M\a)}
	q^{-1/3} \right]
	\right\},
	 %\left[1+{L(qM\a) \over L(M\a)}\right]^{3/2}
	 %\\
	%&+\mathrm{BF}\cdot
	%\int {q \mathrm{d}q \over B}\,f(q)\,\Lambda^2\left({r\a\over B}, qM\a\right)\,
	%{R(qM\a) \over R(M\a)}q^{-1/3}
	%\left[1+{L(qM\a) \over L(M\a)}\right]^{3/2},
\end{align}
where
\begin{equation}
	A=\left[1+{L(qM\a) \over L(M\a)}\right]^{-1/2}, \quad
	B={R(M\a)\over R(qM\a)}\left[1+{L(M\a) \over L(qM\a)}\right]^{-1/2}.
\end{equation}

\section{Examples}

\subsection{Twin Binary}

We have $f(q)=\delta(q-1)$. Thus
\begin{align}
	\Lambda\a(r\a, M\a)
	={1\over 1+\mu(\mathrm{BF}, M\a)}
	\left\{\Lambda^0(r\a, M\a)
	%+\mathrm{BF}\cdot 2^{3/2}\sqrt{2}\left[\Lambda^1(\sqrt{2}\,r\a, M\a)+\Lambda^2(\sqrt{2}\,r\a, M\a)\right].
	+\mu(\mathrm{BF}, M\a)\cdot\sqrt{2}\left[\Lambda^1(\sqrt{2}\,r\a, M\a)+\Lambda^2(\sqrt{2}\,r\a, M\a)\right]
	\right\},
\end{align}
where
\begin{equation}
	\mu(\mathrm{BF}, M\a)=\int \mathrm{d}q\,f(q) {\mathrm{BF}\over{1-\mathrm{BF}}}\left[1+{L(qM\a) \over L(M\a)}\right]^{3/2}
	=2^{3/2}\cdot{\mathrm{BF}\over{1-\mathrm{BF}}}.
\end{equation}

\subsubsection{Same Planets}

If $\Lambda^i(r, M)=Z^i\cdot\delta(r-r_0)$ (all the planets have the same radius),
\begin{align}
	\Lambda\a(r\a, M\a)
	%=\Lambda_0\left[(1-\mathrm{BF})\cdot\delta(r\a-r_0)
	%+2\mathrm{BF}\cdot 2^{3/2}\cdot\delta\left(r\a-{r_0\over\sqrt{2}}\right)\right].
	={1 \over 1+\mu(\mathrm{BF})}
	\left[Z^0 \cdot \delta(r\a-r_0)+(Z^1+Z^2)\cdot \mu(\mathrm{BF})\cdot \delta\left(r\a-{r_0\over\sqrt{2}}\right)\right].
\end{align}
This reproduces Eq.(9) of the draft, after correcting for the smaller number of apparently searchable stars for the diluted planets (factor of $(1/2)^3$). In the limit of $\mathrm{BF}\to1$ ($\mu\to\infty$), the above formula yields $\Lambda\a=(Z^1+Z^2)\,\delta(r\a-{r_0/\sqrt{2}})$.


\subsection{Power Law World}

If we assume 
\begin{equation}
	L(M) \sim M^\alpha \sim R^\alpha,
\end{equation}
we find
\begin{align}
	A=(1+q^\alpha)^{-1/2}, \quad 
	B=q^{-1}(1+q^{-\alpha})^{-1/2},
\end{align}
so
\begin{align}
	\Lambda\a(r\a, M\a)=...
	%&=(1-\mathrm{BF})\cdot\Lambda^0(r\a, M\a)\\
	%&+\mathrm{BF}\cdot 
	%\int \mathrm{d}q\,f(q)
	%(1+q^\alpha)^2 \cdot \Lambda^1\left( r\a\sqrt{1+q^\alpha}, M\a\right)\\
	%&+\mathrm{BF}\cdot
	%\int {\mathrm{d}q}\,f(q)q^{8/3}
	%\sqrt{1+q^{-\alpha}}(1+q^\alpha)^{3/2}
	%\cdot\Lambda^2\left(r\a q\sqrt{1+q^{-\alpha}}, qM\a\right).
\end{align}
We may further assume
\begin{equation}
	f(q)\sim q^\beta, \quad \Lambda(r) \sim r^\gamma
\end{equation}
and keep calculating...


%\acknowledgements
%It was a pleasure to share discussions with Kento Masuda, who pointed us in 
%this direction, and helped clarify that something like this was worth 
%studying.
%It was a pleasure discussing this work with F. Dai and T. Barclay.


\newpage
\bibliographystyle{yahapj}                            
\bibliography{bibliography} 

\end{document}
