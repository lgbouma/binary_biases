\documentclass[12pt,modern]{aastex61}
\usepackage{graphics,graphicx}
\usepackage{hyperref}
\usepackage{amssymb}
\usepackage{amsmath}
\usepackage{comment}

%% Reintroduced the \received and \accepted commands from AASTeX v5.2
%\received{July 1, 2016}
%\revised{September 27, 2016}
%\accepted{\today}
%% Command to document which AAS Journal the manuscript was submitted to.
%% Adds "Submitted to " the arguement.
\submitjournal{AAS journals.}


\shortauthors{Bouma et al.}
\shorttitle{Binarity and Occurrence Rates}

\begin{document}
    
\title{ The effects of binarity on planet occurrence rates measured by transit 
surveys}
%
\correspondingauthor{L. Bouma}
\email{luke@astro.princeton.edu}
%
\author{L. G. Bouma}
\affiliation{
    Department of Astrophysical Sciences,
    Princeton University,
    4 Ivy Lane, Princeton, NJ 08540, USA}
\author{J. N. Winn}
\affiliation{
    Department of Astrophysical Sciences,
    Princeton University,
    4 Ivy Lane, Princeton, NJ 08540, USA}
\author{K. Masuda}
\affiliation{
    Department of Astrophysical Sciences,
    Princeton University,
    4 Ivy Lane, Princeton, NJ 08540, USA}
%
%
\begin{abstract}
%
This derives the equations of the paper.

%
\end{abstract}
%
\keywords{
    methods: data analysis ---
    planets and satellites: detection ---
    surveys}
%
%

\newcommand{\pt}{\theta}
\newcommand{\pa}{\theta_{\rm a}}
\newcommand{\pn}{\theta_0}

\newcommand{\pp}{\mathcal{P}}
\newcommand{\ps}{\mathcal{S}}
\renewcommand{\a}{_{\rm a}}
\newcommand{\s}{_{\rm s}}

\section{Preliminaries}

\subsection{Searchable Distance}

We can detect a signal if 
\begin{equation}
	{\text{signal}\over\text{noise}}\sim
    {\delta_{\rm obs}\over(L/d^2)^{-1/2}},
	\quad \delta_{\rm obs}:\text{observed depth}
\end{equation}
is above some threshold (it would probably make more sense to include the 
duration information, but that would anyway be a trivial extension and so we 
omit it here for brevity). Thus, the maximum searchable distance scales as
\begin{equation}
	d(\delta_{\rm obs}, L_{\rm sys}) \propto \delta_{\rm obs} \cdot L_{\rm 
	sys}^{1/2}.
\end{equation}
We assume that the signal is detected if and only if a given star is searchable. 

Assuming that stars are uniformly distributed in space, the number of 
searchable stars $N\s$ is then proportional to
\begin{equation}
	N\s(\delta_{\rm obs}, L_{\rm sys}) \propto n \delta_{\rm obs}^3 L_{\rm 
	sys}^{3/2},
\end{equation}
where $n$ is the volume density of {\it e.g.}, single star, or binary systems.
We neglect the dependence of $n$ on stellar type.

\subsection{Relation between Apparent and Actual Stellar Properties}

We assume that the apparent properties of an unresolved binary are the same as 
those of the primary:
\begin{equation}
	M\a=M_1, \quad R\a=R_1, ...
\end{equation}
We also assume the stellar radius and luminosity is uniquely related to the 
stellar mass.

Given these assumptions, the total luminosity of a system is
\begin{equation}
	L_{\rm sys}=L_1+L_2=L(M\a)+L(qM\a),
\end{equation}
where $q=M_2/M_1$. Note that $L_{\rm sys}$ is the true value (because $M\a$ is 
the true primary mass), while an observer would estimate a system luminosity 
of $L(M\a)$, based on the apparent stellar parameters. We presume the 
observer has no a priori knowledge of the distance to a given system~--~they 
only measure a flux, assume a stellar mass $M_a$, and use relations for 
$L(M_a)$ and $R(M_a)$ to estimate system properties.

\subsection{Apparent Number of Searchable Stars}

Given the apparent signal $\delta_{\rm obs}$ and stellar mass $M\a$, the 
maximum searchable distance for singles and binaries are proportional to 
$\delta_{\rm obs}\cdot L(M\a)^{1/2}$ and $\delta_{\rm obs}\cdot 
[L(M\a)+L(qM\a)]^{1/2}$. Thus, 
the apparent number of searchable stars (i.e. points in the sky), which will 
be selected by an ignorant observer, can be written 
\begin{align}
	\notag
	N_{\rm s,a}(\delta_{\rm obs}, M\a)
	%&=n_{\rm s,0}\delta_{\rm obs}^3L(M\a)^{3/2} + n_{\rm b,0}\delta_{\rm 
	%obs}^3 \cdot [L(M\a)+L(qM\a)]^{3/2}\\
	&\propto n_{\rm s}\delta_{\rm obs}^3 L(M\a)^{3/2}\left[ 1+\int 
	\mathrm{d}q\,f(q) 
	{\mathrm{BF}\over{1-\mathrm{BF}}}\left(1+{L(qM\a) \over 
	L(M\a)}\right)^{3/2}\right]\\
	N_{\rm s,a}(\delta_{\rm obs}, M\a)
    &\equiv N\s^0(\delta_{\rm obs}, L(M\a)) \left[1+\mu(\mathrm{BF}, 
	M\a)\right],
\end{align}
where $N\s^0$ is the number of searchable singles (this agrees with the actual 
value), $f(q)$ is the binary mass ratio distribution, $\mathrm{BF}=n_{\rm 
b}/(n_{\rm s}+n_{\rm b})$ is the binary fraction in a volume-limited sample, 
and $n_s$ is the number density of singles in a volume-limited sample.
Proportionality constants subsumed into the definition of $N_s^0$ include 
{\it e.g.}, the telescope area and the survey duration.


\section{Apparent Occurrence Rate --- General Formula}

A group of astronomers wants to measure the mean number of planets of a 
certain type per star of a certain type.
They observe a set points on the sky and detect $N_{\rm det}$ planets that 
appear to be of the desired class.
They then choose the stars (among those initially selected) around which the 
planets of interest appeared to be searchable.
Finally they compute the apparent occurrence,
\begin{equation}
\Lambda\a(\pp\a, \ps\a) = \frac{N_{\rm det}(\pp\a, \ps\a)}{N_{\rm s,a}(\pp\a, 
    \ps\a)} \times \frac{1}{p_{\rm tra}(\pp\a, \ps\a)}.
\end{equation}
where $\pp\a$, $\ps\a$ are the apparent planetary/stellar parameters.

In the presence of dilution, planets with $(\pp\a, \ps\a)$ are associated with 
systems of many different planetary and stellar properties, so $N_{\rm 
det}(\pp\a, \ps\a)$ is 
given by the convolution of the true occurrence, $\Lambda(\pp, \ps)$, and 
number of searchable stars that give $(\pp\a, \ps\a)$ when the true system 
actually has $(\pp, \ps)$, $\mathcal{N}(\pp\a, \ps\a; \pp, \ps)$:
\begin{equation}
	N_{\rm det}(\pp\a, \ps\a)=\sum_i N_{\rm det}^i(\pp\a, \ps\a)
	=\sum_i \int \mathrm{d}\pp\mathrm{d}\ps\,
	%\mathcal{P}_{\rm tra}(\pp\a, \ps\a; \pt)
	\mathcal{N}_{\rm s}^i(\pp\a,\ps\a; \pp, \ps)
	\cdot\Lambda^i(\pp, \ps)\cdot p_{\rm tra}(\pp, \ps),
\end{equation}
where $i$ specifies the type of true host stars (0: single, 1: primary, 2: 
secondary).

So the problem reduces to the evaluation of
\begin{equation}
	\mathcal{N}_{\rm s}^i(\pp\a,\ps\a; \pp, \ps)
\end{equation}
for planets around single stars, primaries in binaries, and secondaries in binaries. 

\section{Evaluation of 
    $\mathcal{N}_{\rm \MakeLowercase{s}}^{\MakeLowercase{i}}$}

Let us explicitly write $\pp=r$ and $\ps=M$; $R$ and $L$ are uniquely 
determined from the assumed mass--radius--luminosity relation. We neglect the 
dependence on planetary orbital period.
We proceed by evaluating
\begin{align}
N_{\rm det}(r_a, M_a) &=
\sum_i N_{\rm det}^i(r_a, M_a) \\
N_{\rm det}(r_a, M_a)
&=
\sum_i \int \mathrm{d}r \mathrm{d}M \,
\mathcal{N}_{\rm s}^i(r_a,M_a; r,M)
\cdot\Lambda^i(r,M) \cdot p_{\rm tra}(r,M),
\end{align}
term by term.

\subsection{Single Stars}

For $i=0$, 
\begin{equation}
	\mathcal{N}_{\rm s}^0(r\a,M\a; r,M)
	=\delta(r\a-r)\delta(M\a-M) N\s^0(r,M),
\end{equation}
so
\begin{equation}
	N_{\rm det}^0(r\a, M\a)=N\s^0(r\a,M\a)\cdot \Lambda^0(r\a, 
	M\a) \cdot p_{\rm tra}(r\a, M\a).
\end{equation}
If all the stars are singles, this yields
\begin{equation}
	\Lambda\a(r\a, M\a)=\Lambda^0(r\a, M\a),
\end{equation}
as expected (now $\mu=0$ and $N_{\rm s,a}=N\s^0$) --- the true occurrence is 
recovered.

%TODO: PROCEED FROM HERE

\subsection{Primaries in Binaries}

Since we assume $\ps\a=\ps_1$,
\begin{equation}
	\mathcal{N}_{\rm s}^1(r_a, M_a; r, M) \propto \delta(M_a-M).
\end{equation}
In this case, $\mathcal{N}_{\rm s}^1$ is non-zero only at 
$r_a=R_a\sqrt{\delta_{\rm obs}}$, 
and the observed depth is
\begin{equation}
	\delta_{\rm obs}
	= \left[{r\over R(M_a)}\right]^2\times {L(M_a) \over L_{\rm sys}(M_a, q)}.
   \label{eq:delta_obs_primaries} 
\end{equation}
The normalization of $\mathcal{N}_{\rm s}^1$ is given by the number of 
binaries that are searchable for a signal $\delta_{\rm obs}$:
\begin{equation}
	N_{\rm s}^0(\delta_{\rm obs}, 
	L(M_a))\cdot\mu(\mathrm{BF},M_a).
\end{equation}
Thus, the number of primaries with apparent parameters $(r_a,M_a)$ given the 
true parameters $(r,M)$ is
\begin{equation}
	\mathcal{N}_{\rm s}^1(r_a, M_a; r, M)
	=\int \mathrm{d}q\,f(q)\mathcal{N}_{{\rm s}, q}^1(r_a, M_a; r, M; q),
\end{equation}
where $f(q)$ is the binary mass ratio distribution and
\begin{align}
	\notag
	\mathcal{N}_{{\rm s}, q}^1(r_a, M_a; r, M; q)
	&=N_{\rm s}^0(\delta_{\rm obs}, L(M_a))\cdot\mu(\mathrm{BF},M_a)\\
	&\times\delta \left(r_a-r\sqrt{{L(M_a) \over L_{\rm sys}(M_a, 
	q)}}\right)\delta(M_a-M).
\label{eq:N_s_1}
\end{align}
Note that while the observed depth is a function of the true parameters 
(Eq.~\ref{eq:delta_obs_primaries}), since we are counting 
detected planets of a given {\it apparent} size, we will not need to worry 
about this dependence.


\subsection{Secondaries in Binaries}

In this case, $M=qM_1=qM_a$, so
\begin{equation}
	\mathcal{N}_{\rm s}^2(r_a, M_a; r, M)
	\propto \delta\left(M_a-{M\over q}\right).
	%=\mathcal{N}_{\rm s}^2(r_a, M_a, R_a; r, qM_a, R(qM_a)).
\end{equation}
Again $\mathcal{N}_{\rm s}^2$ is non-zero only at $r_a=R_a\sqrt{\delta_{\rm 
obs}}$, but this 
time
\begin{equation}
	\delta_{\rm obs} = \left[{r\over R(qM_a)}\right]^2 \times {L(qM_a) 
	\over L_{\rm sys}(M_a, q)}.
\end{equation}
The normalization remains the same as the previous case (we are counting the 
searchable stars at a given observed depth $\delta_{\rm obs}$, total 
luminosity of the binary is the same).
Thus,
\begin{equation}
	\mathcal{N}_{\rm s}^2(r_a, M_a; r, M)
	=\int \mathrm{d}q\,f(q)\mathcal{N}_{{\rm s}, q}^2(r_a, M_a; r, M; q),
\end{equation}
where
\begin{align}
	\notag
	\mathcal{N}_{\rm s}^2(r_a, M_a; r, M; q)
	%&=N_{\rm s}(\delta_{\rm obs}, L(M_a)) \left[L_{\rm sys}(M_a, q) \over 
	%L(M_a) 
	%\right]^{3/2}\\
	&=N_{\rm s}^0(\delta_{\rm obs}, L(M_a))\cdot\mu(\mathrm{BF},M_a)\\
	&\times \delta \left(r_a-r\sqrt{ \left[{R(M_a)\over R(qM_a)}\right]^2 
	{L(qM_a) 
	\over L_{\rm sys}(M_a, q)} }\right)\delta\left(M_a-{M\over q}\right).
\end{align}

\section{Result}

\subsection{Marginalization over the True Properties}

Let's integrate out $\pp$ and $\ps$.

\begin{align}
	N_{\rm det}^0(r\a, M\a)
	&=\int\mathrm{d}r\mathrm{d}M\,\mathcal{N}_{\rm s}^0(r\a, M\a; r, M)
	\cdot\Lambda^0(r, M) \cdot p_{\rm tra}(M)\\
	&=\int\mathrm{d}r\mathrm{d}M\, N\s^0(\delta_{\rm obs}, 
	L(M\a))\delta(r\a-r)\delta(M\a-M)
	\cdot\Lambda^0(r, M) \cdot p_{\rm tra}(M)\\
	&=N\s^0(\delta_{\rm obs}, L(M\a))\cdot\Lambda^0(r\a, M\a) \cdot p_{\rm 
	tra}(M\a).
\end{align}

\begin{align}
	N_{\rm det}^1(r\a, M\a)
	&=\int\mathrm{d}r\mathrm{d}M\,\mathcal{N}_{\rm s}^1(r\a, M\a; r, M)
	\cdot\Lambda^1(r, M) \cdot p_{\rm tra}(M)\\
	&=\int \mathrm{d}q\,f(q)\int\mathrm{d}r\mathrm{d}M\,\mathcal{N}_{{\rm s}, q}^1(r\a, M\a; r, M; q)\cdot\Lambda^1(r, M) \cdot p_{\rm tra}(M)\\
	&=N\s^0(\delta_{\rm obs}, L(M\a))\cdot p_{\rm tra}(M\a) \cdot
	\mu(\mathrm{BF}, M\a) \int {\mathrm{d}q \over 
	\alpha}\,f(q)\,\Lambda^1(r_a/\alpha, M\a),
	%\left[L_{\rm sys}(M_a, q) \over L(M_a) \right]^{3/2},
\end{align}
where
\begin{equation}
	\alpha(q, M\a)=\sqrt{L(M\a) \over L_{\rm sys}(M\a, q)}.
\end{equation}

Finally,
\begin{align}
	N_{\rm det}^2(r\a, M\a)
	&=\int\mathrm{d}r\mathrm{d}M\,\mathcal{N}_{\rm s}^2(r\a, M\a; r, M)
	\cdot\Lambda^2(r, M) \cdot p_{\rm tra}(M)
    \\
	&=\int \mathrm{d}q\,f(q)\int\mathrm{d}r\mathrm{d}M\,\mathcal{N}_{{\rm s}, 
	q}^2(r\a, M\a; r, M; q)\cdot\Lambda^2(r, M) \cdot p_{\rm tra}(M)
    \\
    &= N\s^0(\delta_{\rm obs},L(M\a)) \mu({\rm BF},M\a)
    \int {\rm d}r{\rm d}q\, q f(q) \delta(r_a - r\beta) \Lambda^2(r,qM\a) 
    p_{\rm tra}(qM\a) \\
	&=%N_{\rm s}(\delta_{\rm obs}, L(M_a)
	N\s^0(\delta_{\rm obs}, L(M\a))\cdot\mu(\mathrm{BF}, M\a)
	\int {q \mathrm{d}q \over \beta}\,f(q)\,\Lambda^2(r_a/\beta, qM\a)\,p_{\rm 
	tra}(qM\a),
	%\left[L_{\rm sys}(M_a, q) \over L(M_a) \right]^{3/2},
\end{align}
where
\begin{equation}
	\beta(q, M\a)={R(M\a)\over R(qM\a)}\sqrt{{L(qM\a) \over L_{\rm sys}(M\a, q)} }.
\end{equation}

\subsection{Final Formula}

Note again that the denominators of $\Lambda\a$ are the same for singles, 
primaries, and secondaries: $N_{\rm s,a}(\delta_{\rm obs}, M\a)$ and $p_{\rm 
tra}(M\a)$. This is because we are calculating the occurrence at the same 
apparent planet/star properties, and the observer can never distinguish 
binaries from singles (and adopt the primary properties for binaries).
Using the results above, the apparent occurrence,
\begin{align}
\Lambda\a(r\a,M\a) &= 
    \frac{1}{N_{\rm s,a}(r\a,M\a) p_{\rm tra}(r\a,M\a)} \times
    \sum_i N_{\rm det}^i (r\a,M\a),
\end{align}
evaluates to
\begin{align}
\notag
\Lambda\a(r\a,M\a) &= {1\over 1+\mu(\mathrm{BF}, M\a)}\times
   \left\{ \Lambda^0(r\a, M\a)+ 
\mu(\mathrm{BF}, M\a)
\left[ \int {\mathrm{d}q \over \alpha}\,f(q)\,\Lambda^1\left({r\a\over 
    \alpha}, 
M\a\right)\,
\right.   
   \right. \\
& \quad\quad\quad\quad\quad \left.\left.
+\int {q \mathrm{d}q \over \beta}\,f(q)\,\Lambda^2\left({r\a\over \beta}, 
qM\a\right)\,
{R(qM\a) \over R(M\a)}
q^{-1/3} \right]	\right\}.  
\end{align}


\section{Examples}

\subsection{Twin Binary}

We have $f(q)=\delta(q-1)$. Thus
\begin{align}
	\Lambda\a(r\a, M\a)
	={1\over 1+\mu(\mathrm{BF}, M\a)}
	\left\{\Lambda^0(r\a, M\a)
	%+\mathrm{BF}\cdot 2^{3/2}\sqrt{2}\left[\Lambda^1(\sqrt{2}\,r\a, M\a)+\Lambda^2(\sqrt{2}\,r\a, M\a)\right].
	+\mu(\mathrm{BF}, M\a)\cdot\sqrt{2}\left[\Lambda^1(\sqrt{2}\,r\a, M\a)+\Lambda^2(\sqrt{2}\,r\a, M\a)\right]
	\right\},
\end{align}
where
\begin{equation}
	\mu(\mathrm{BF}, M\a)=\int \mathrm{d}q\,f(q) {\mathrm{BF}\over{1-\mathrm{BF}}}\left[1+{L(qM\a) \over L(M\a)}\right]^{3/2}
	=2^{3/2}\cdot{\mathrm{BF}\over{1-\mathrm{BF}}}.
\end{equation}

\subsubsection{Same Planets}

If $\Lambda^i(r, M)=Z^i\cdot\delta(r-r_0)$ (all the planets have the same radius),
\begin{align}
	\Lambda\a(r\a, M\a)
	%=\Lambda_0\left[(1-\mathrm{BF})\cdot\delta(r\a-r_0)
	%+2\mathrm{BF}\cdot 2^{3/2}\cdot\delta\left(r\a-{r_0\over\sqrt{2}}\right)\right].
	={1 \over 1+\mu(\mathrm{BF})}
	\left[Z^0 \cdot \delta(r\a-r_0)+(Z^1+Z^2)\cdot \mu(\mathrm{BF})\cdot \delta\left(r\a-{r_0\over\sqrt{2}}\right)\right].
\end{align}
This reproduces Eq.(9) of the draft, after correcting for the smaller number of apparently searchable stars for the diluted planets (factor of $(1/2)^3$). In the limit of $\mathrm{BF}\to1$ ($\mu\to\infty$), the above formula yields $\Lambda\a=(Z^1+Z^2)\,\delta(r\a-{r_0/\sqrt{2}})$.


\subsection{Power Law World}

If we assume 
\begin{equation}
	L(M) \sim M^\alpha \sim R^\alpha,
\end{equation}
we find
\begin{align}
	A=(1+q^\alpha)^{-1/2}, \quad 
	B=q^{-1}(1+q^{-\alpha})^{-1/2},
\end{align}
so
\begin{align}
	\Lambda\a(r\a, M\a)=...
	%&=(1-\mathrm{BF})\cdot\Lambda^0(r\a, M\a)\\
	%&+\mathrm{BF}\cdot 
	%\int \mathrm{d}q\,f(q)
	%(1+q^\alpha)^2 \cdot \Lambda^1\left( r\a\sqrt{1+q^\alpha}, M\a\right)\\
	%&+\mathrm{BF}\cdot
	%\int {\mathrm{d}q}\,f(q)q^{8/3}
	%\sqrt{1+q^{-\alpha}}(1+q^\alpha)^{3/2}
	%\cdot\Lambda^2\left(r\a q\sqrt{1+q^{-\alpha}}, qM\a\right).
\end{align}
We may further assume
\begin{equation}
	f(q)\sim q^\beta, \quad \Lambda(r) \sim r^\gamma
\end{equation}
and keep calculating...


%\acknowledgements
%It was a pleasure to share discussions with Kento Masuda, who pointed us in 
%this direction, and helped clarify that something like this was worth 
%studying.
%It was a pleasure discussing this work with F. Dai and T. Barclay.


\newpage
\bibliographystyle{yahapj}                            
\bibliography{bibliography} 

\end{document}
