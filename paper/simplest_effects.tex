\section{Idealized Models of Transit Surveys}

\subsection{Model \#1: fixed stars, fixed planets, twin binaries}
\label{sec:model_1}

Since the effects of binarity are most pronounced when the two components are 
similar, we begin by considering a universe in which all planets are 
identical, and all stars are identical except that some fraction of them exist 
in binaries.

Expressed mathematically, from Eqs.~\ref{eq:rate_density_marginalized}
and~\ref{eq:rate_density_shape} the occurrence rate density at a planet radius 
$r$ can be written
\begin{equation}
\Gamma(r) = \delta(r_p) \times
\frac{
    N_0 Z_0 +
    N_1 Z_1 +
    N_2 Z_2
}{N_{\rm tot}},
\label{eq:model_1_rate_density}
\end{equation}
where $\delta(r_p)$ is the Dirac delta function, zero except at the true 
planet radius, $r_p$.
The occurrence rate over any interval that includes $r_p$ is then
\begin{equation}
\Lambda|_{r_p} = \frac{
    N_0 Z_0 +
    N_1 Z_1 +
    N_2 Z_2 
}{N_{\rm tot}},
\label{eq:model_1_rate}
\end{equation}
and the rate is zero over intervals that do not include $r_p$.

We return to our group of binarity-ignoring astronomers. They do 
not know the true planet population~--~they would like to discover it!
In their signal-to-noise limited transit survey, they select stars 
that they think can yield transit detections.
Since the noise is Poissonian, they assume
\begin{equation}
\frac{{\rm signal}}{{\rm noise}}
\propto \frac{(r/R)^2}{F^{-1/2}}
\propto F^{1/2}
\propto L_{\rm sys}^{1/2} d^{-1}, \quad ({\rm incorrectly\ assumed})
\end{equation}
for $R$ the (constant) stellar radius, $F$ the photon flux, $L_{\rm sys}$ the 
luminosity of a system, and $d$ its distance from us.
At fixed planet radius, semimajor axis, stellar radius, and stellar luminosity,
a constant signal-to-noise floor yields a maximum detectable 
distance (Pepper et al 2003, Pepper and Gaudi 2005).
The maximum distance out to which our binarity-ignoring astronomers 
select stars, $d_{\rm sel}$, thus scales as $L_{\rm sys}^{1/2}$.

The single stars have luminosity $L_1$, and the twin binaries have luminosity 
$2L_1$.
Thus the twin binaries are selected out to a distance $\sqrt{2}$ times 
that of single stars.
This is a bad move; in reality the transit signal for any planet in a 
twin binary will be diluted by a factor of two:
\begin{equation}
\frac{{\rm signal}}{{\rm noise}}
\propto \frac{\mathcal{D} (r/R)^2}{F^{-1/2}}
\propto \mathcal{D} F^{1/2}
\propto \mathcal{D} L_{\rm sys}^{1/2} d^{-1}, \quad ({\rm true})
\end{equation}
where the dilution is $\mathcal{D} \equiv L_{\rm host}/L_{\rm sys}$, for 
$L_{\rm host}$ the the planet host's luminosity.
This means that the actual maximum searchable distance for binaries is 
$1/\sqrt{2}\times$ that of single stars.
The situation is illustrated in Fig.~\ref{fig:model_1_volumes}: only one in 
eight selected stars in binaries are truly searchable.

\paragraph{What do the observers ignoring binarity infer?} 
The binarity-ignoring observers assume that every point on the sky with flux 
above some minimum are searchable.
They measure a detected planet rate density, and infer an apparent rate 
density by correcting for $p_{\rm tra}$.
The apparent rate density is
\begin{equation}
\Gamma_a(r) = 
\delta(r_p) Z_0 \frac{N_0}{N_0+N_1}  +
\delta\left(\frac{r_p}{\sqrt{2}}\right) 
(Z_1 p_{{\rm det},1} + Z_2 p_{{\rm det},2}) \frac{N_1}{N_0+N_1},
\label{eq:model_1_apparent_rate_density}
\end{equation}
where $p_{{\rm det},1}$ ($p_{{\rm det},2}$) is the probability that a selected 
primary (secondary) is searchable.
For this example, $p_{{\rm det},1}=p_{{\rm det},2}=1/8$.

The true rate density (Eq.~\ref{eq:model_1_rate_density}) and the
apparent rate density (Eq.~\ref{eq:model_1_apparent_rate_density})
differ in that
\begin{enumerate}
\item The total number of selected stars, $N_{\rm tot} = N_0+N_1+N_2$, was 
miscounted.
%
\item The detection efficiency was incorrectly assumed to be $1$ for all 
selected stars. In reality, only one in eight binaries were searchable.
%
\item The inferred radii in binary systems are all $\sqrt{2}$ too small.
\end{enumerate}

We show the resulting occurrence rates over bins in planet radius in 
Fig.~\ref{fig:errcases_model_1}.




%NOTE: YOU CAN BUTCHER THE FUCK OUT OF THIS SECTION. JNW WANTS EACH CASE TO BE 
%EXPLAINED, INDIVIDUALLY. DROP ALL THE "i" SUBSCRIPTS B/C THEY ARE APPARENTLY 
%CONFUSING.


To express the rate density of detected planets, $\hat{\Gamma} = \sum 
Q_i\Gamma_i$, we need the detection efficiencies for each system type, which 
are products of the geometric and selection probabilities:
\begin{align}
Q_i(\vec{x}) &= Q_{g,i}(\vec{x}) Q_{c,i}(\vec{x}),\quad {\rm where}\ 
\vec{x}=(r,R,a).
\label{eq:general_detection_efficiency}
\end{align}
Similar to Pepper et al. (2003), but in a new context, we take $Q_c$ as the 
ratio of the number of stars that were searchable to the number of stars that 
were selected.
Assuming a homogeneous distribution of stars, this gives
\begin{equation}
Q_{c,i}(\vec{x}) = \left(
\frac{d_{{\rm det},i}(\vec{x})}{d_{\rm sel}(\vec{x})}
\right)^3,
\end{equation}
for $d_{\rm sel}$ the maximum distance to which surveyed stars are selected, 
and $d_{{\rm det},i}$ the maximum distance to which planets can actually be 
detected about the $i^{\rm th}$ system type.
Note that $d_{\rm sel} \geq d_{{\rm det},i}$.
In a signal-to-noise limited transit survey in which the observer does not 
know which stars are binaries, 
\begin{equation}
d_{\rm sel} \propto (r/R)^2 (L_{\rm sys} T_{\rm dur} A N_{\rm tra})^{1/2},
\end{equation}
for $L_{\rm sys}=L_1(1+\ell)$ the system luminosity, $T_{\rm dur}$ the 
transit duration, $A$ the detector area, and $N_{\rm tra}$ 
the number of observed transits.
However,
\begin{equation}
d_{{\rm det},i} \propto \mathcal{D}_i(r/R)^2 (L_{\rm sys} T_{\rm dur} A N_{\rm 
    tra})^{1/2},
\label{eq:d_det_i}
\end{equation}
for the dilution $\mathcal{D}_i$ given by
\begin{align}
\mathcal{D}_i
&=
\left.
\begin{cases}
1 & \text{for } i=0,\ {\rm single} \\
L_1 / L_{\rm sys} = (1+\ell)^{-1}, & \text{for } i=1,\ {\rm primary} \\
\ell L_1 / L_{\rm sys} = (1 + \ell^{-1})^{-1}, & 
\text{for } i=2,\ {\rm secondary},
\end{cases}
\right.
\label{eq:dilution}
\end{align}
where the light ratio $\ell$ of a given binary is defined as the ratio of 
the luminosity of the secondary to the primary.

The maximum detectable distance to single stars is assumed to be known, and so 
$d_{{\rm sel},0} = d_{{\rm det},0}$.
This means our naive astronomer assumes the fraction of their selected stars 
which are searchable is 1 (this is their ``assumed completeness'').
For binary systems there is a necessary incompleteness, and combining 
Eqs.~\ref{eq:general_detection_efficiency} through~\ref{eq:dilution} yields
\begin{align}
Q_0 &= Q_{g,0}Q_{c,0} = Q_{g,0} \label{eq:detection_efficiency_0}\\
Q_1 &= Q_{g,1}Q_{c,1} = Q_{g,0} (1+q^\alpha)^{-3} \\
Q_2 &= Q_{g,2}Q_{c,2} = Q_{g,0} q^{2/3} (1+q^{-\alpha})^{-3} q^{-5}, 
\label{eq:detection_efficiency_2}
\end{align}
for $Q_{g,0}=R/a$, the transit probability in single star systems.
In Eq.~\ref{eq:detection_efficiency_2}, we have assumed a stellar 
mass-luminosity-radius relation: $R\propto M \propto L^{1/\alpha}$.
The geometric transit probability is $Q_{g,2} = Q_{g,0}q^{2/3}$, and the 
completeness is the latter term of Eq.~\ref{eq:detection_efficiency_2}.
For $q=1$, the $q^{2/3}$ and $q^{-5}$ evaluate to unity, but they will later 
become relevant.

Summarizing, we have written the rate density for each system type
(Eq.~\ref{eq:model1_occ_rate_density}) and the detection efficiency for each 
system type 
(Eq.~\ref{eq:detection_efficiency_0}-\ref{eq:detection_efficiency_2}),
and so have fully specified the rate density of detected planets, 
in addition to the true rate density.




\paragraph{Correction to inferred rate density and inferred rate}

Define a rate density correction factor, $X_\Gamma$, as the ratio of the 
apparent to true rate densities:
\begin{equation}
X_\Gamma \equiv \frac{\Gamma_a}{\Gamma}.
\end{equation}
This factor can be a function of whatever parameters $\Gamma_a$ and $\Gamma$ 
depend on; in this study, the planet radius is most relevant.
For the twin-binaries model,
\begin{equation}
X_\Gamma(r)
=
\frac{w_a \Lambda_0\delta^3(r_p) + 
    w_b(\Lambda_1 Q_{c,1} + \Lambda_2 Q_{c,2}) \delta^3(r_p/\sqrt{2})  }
{(w_0\Lambda_0 + w_1\Lambda_1 + w_2\Lambda_2)\delta^3(r_p)}
\label{eq:model1_correction}
\end{equation}
where $\delta^3(r_p)$ is shorthand for $\delta^3(r-r_p,R-R_\star,a-a_p)$.

If we take the rates $\Lambda_i$ to be equal, applying the definitions of 
the weights gives a rate density correction factor at $r=r_p$ of
$X_\Gamma(r_p) = (1+\mu)^{-1}$, where 
\begin{align}
\mu \equiv \frac{N_1}{N_0} &=
\frac{n_b}{n_s} \left(\frac{d_{\rm sel,b}}{d_{\rm sel,s}}\right)^3 = 
\frac{{\rm BF}}{1-{\rm BF}} (1+\ell)^{3/2},
\label{eq:mu_definition}
\end{align}
for $n_b$ and $n_s$ the number density of binaries and singles in a 
volume limited sample.
Using Raghavan et al. (2010)'s $0.7-1.3M_\odot$ multiplicity fraction as our 
binary fraction\footnote{
    The binary fraction is the fraction of systems in a volume-limited sample 
    that 
    are binary. It is equivalent to the multiplicity fraction if there are no 
    triple, quadruple, or higher order multiples. In that case, ${\rm BF} = 
    n_b / 
    (n_s+n_b)$.
}, we set ${\rm BF}=0.44$.
The resulting correction to the rate density is $X_\Gamma(r_p) \approx 0.31$. 
The correction at $r_p/\sqrt{2}$ is infinite.
The numerical realization of this model agrees with these analytic values, and 
its output is shown in Fig~\ref{fig:errcases_model_1}.
%beta = 2.2223355980148636
If instead we assume that $\Lambda_0 = \Lambda_1$, but that $\Lambda_2=0$, we 
find 
$X_\Gamma(r_p) = (1+2\mu)/(1+\mu)^2$.
Taking the same binary fraction, this evaluates to $X_\Gamma(r_p)\approx 0.53$.
Since the correction to the rate is equal to that of rate density, at 
$r=r_p$, the occurrence rate is underestimated by a factor of roughly 2 to 3.

%Note that a correction to the inferred rate, $X_\Lambda$, can be 
%defined analogously:
%\begin{equation}
%X_\Lambda \equiv \frac{\Lambda_a}{\Lambda}.
%\end{equation}
%For this twin binary model, the correction to the rate is the same as that to 
%the rate density.
