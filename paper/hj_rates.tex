%% \begin{deluxetable}{} command tell LaTeX how many columns
%% there are and how to align them.
\begin{deluxetable}{cccc}
    
%% Keep a portrait orientation

%% Over-ride the default font size
%% Use Default (12pt)

%% Use \tablewidth{?pt} to over-ride the default table width.
%% If you are unhappy with the default look at the end of the
%% *.log file to see what the default was set at before adjusting
%% this value.

%% This is the title of the table.
\caption{Occurrence rates of hot Jupiters (HJs) about FGK dwarfs, as measured 
by radial velocity and transit surveys.}
\label{tab:hj_rates}

%% This command over-rides LaTeX's natural table count
%% and replaces it with this number.  LaTeX will increment 
%% all other tables after this table based on this number
\tablenum{1}

%% The \tablehead gives provides the column headers.  It
%% is currently set up so that the column labels are on the
%% top line and the units surrounded by ()s are in the 
%% bottom line.  You may add more header information by writing
%% another line between these lines. For each column that requries
%% extra information be sure to include a \colhead{text} command
%% and remember to end any extra lines with \\ and include the 
%% correct number of &s.
\tablehead{\colhead{Reference} & \colhead{HJs per thousand stars} & 
\colhead{HJ Definition} 
%& 
%\colhead{} \\ 
%    \colhead{} & \colhead{(planets per thousand stars)} & \colhead{} & 
%    \colhead{}
} 

%% All data must appear between the \startdata and \enddata commands
\startdata
\citet{marcy_observed_2005}
    & 12$\pm$2 & $a<0.1\,{\rm AU}; P\lesssim10\,{\rm day}$ \\
\citet{cumming_keck_2008} & 15$\pm$6 & -- \\
\citet{mayor_harps_2011} & 8.9$\pm$3.6 & -- \\
\citet{wright_frequency_2012} & 12.0$\pm$3.8 & -- \\
\citet{gould_frequency_2006} & $3.1^{+4.3}_{-1.8}$ & $P<5\,{\rm day}$ \\
\citet{bayliss_frequency_2011} & $10^{+27}_{-8}$ & $P<10\,{\rm day}$ \\
\citet{howard_planet_2012} 
    & 4$\pm$1 & $P<10\,{\rm day}; r_p=8-32r_\oplus$; solar 
    subset\tablenotemark{a} 
\\
-- & 5$\pm$1 & solar subset extended to $Kp<16$ \\
-- & 7.6$\pm$1.3 & solar subset extended to $r_p>5.6r_\oplus$. \\
\citet{moutou_corot_2013} 
    & 10$\pm$3 & {\it CoRoT} average; $P\lesssim 10\,{\rm day}$, 
    $r_p>4r_\oplus$  \\
Petigura et al. (2018, in prep) & $5.7^{+1.4}_{-1.2}$ &
    $r_p=8-24r_\oplus$; $P=1-10\,{\rm day}$; CKS stars\tablenotemark{b} \\
Santerne et al. (2018, in prep) & 9.5$\pm$2.6 & {\it CoRoT} galactic center \\
-- & 11.2$\pm$3.1 & {\it CoRoT} anti-center \\
\enddata

%% Include any \tablenotetext{key}{text}, \tablerefs{ref list},
%% or \tablecomments{text} between the \enddata and 
%% \end{deluxetable} commands

%% General table comment marker
\tablecomments{
    The first four studies use data from radial velocity surveys; the rest
    are based on transit surveys. Many of these surveys selected different 
    stellar samples. ``--'' denotes ``same as above''.
}
\tablenotetext{a}{
    \citet{howard_planet_2012}'s ``solar subset'' was defined as {\it 
    Kepler}-observed stars with $4100\,{\rm K}<T_{\rm eff}<6100\,{\rm K}$, $Kp 
    <15$, $4.0 < \log g < 4.9$. They required signal to noise $>10$ for planet 
    detection.
    }
\tablenotetext{b}{
    Petigura et al. (2018, in prep)'s planet sample includes all KOIs with 
    $Kp<14.2$, with a statistically insignificant number of fainter stars with 
    HZ planets and 
    multiple transiting planets.
    Their stellar sample begins with \citet{mathur_revised_2017}'s catalog of 
    199991 {\it Kepler}-observed stars.
    Successive cuts are: $Kp<14.2\,{\rm mag}$, $T_{\rm eff}=4700-6500\,{\rm 
    K}$, and $\log g = 3.9-5.0\,{\rm dex}$, leaving $33020$ stars.
}   
\end{deluxetable}
