\section{Discussion}
\label{sec:discussion}

\paragraph{How has binarity been considered in occurrence rate measurements?}
Binarity introduces systematic uncertainty to star and planet counts, and also 
to estimates of pipeline completeness.
In spite of this fact, stellar multiplicity has mostly been ignored in 
calculations of planet occurrence rates using transit survey data\footnote{
    An online list of occurrence rate papers is maintained at 
    \url{https://exoplanetarchive.ipac.caltech.edu/docs/occurrence_rate_papers.html}
}~({\it 
e.g.}, Howard et al. 2012, Fressin et al., 2013, Foreman-Mackey et 
al., 2014, Dressing \& Charbonneau 2015, and Burke et al. 2015).
For {\it Kepler} occurrence rates specifically, it seems that no one has yet 
carefully assessed the magnitude of the issue.
While we do not resolve the problem, we do suggest
the approximate scale of the necessary corrections in a survey-independent 
manner.

Of course, on a system-by-system level stellar multiplicity affects the 
interpretation of planet candidates. High resolution imaging 
campaigns have measured the multiplicity of almost all {\it Kepler}\ Objects 
of Interest 
(Howell et al. 2011; Adams et al. 2012, 2013; Horch et al. 2012, 2014; 
Lillo-Box et al. 2012, 2014; Dressing et al. 2014; Law et al. 2014; Cartier et 
al. 2015; Everett et al. 2015; Gilliland et al. 2015; Wang et al. 2015a, 
2015b; Baranec et al. 2016).
The results of these programs have been collected by Furlan et al. 
(2017), and they represent an important advance in understanding the KOI 
sample's multiplicity statistics.
In particular, they can be immediately applied to rectify binarity's effects 
on the mass-radius diagram (Furlan \& Howell 2017).

The high resolution imaging campaign is also beginning to connect with
occurrence rate calculations.
The most recent rate studies have used Furlan et al. (2017)'s 
catalog to test the effects of removing KOI hosts with known companions, which 
helps reduce contamination in the ``numerator'' of 
the occurrence rate (Fulton et al. 2017, Petigura et al., in prep).
However, without an understanding of the multiplicity statistics of the 
non-KOI stars, the true completeness, and thus the true 
occurrence rates, will remain biased.
The first-order correction that we suggest, given the impracticality of 
performing high-resolution imaging of every selected star in a transit survey,
is to model the detection pipeline's efficiency while accounting for 
binarity.
%Simply put~--~do not assume binarity can be ignored, and take the relevant 
%systematic uncertainties about the selected stars into account.
For {\it Kepler}, this would require high resolution imaging of a
comparison sample of non-KOI host stars. If the associated multiplicity 
statistics are then included in a model of the pipeline's detection 
efficiency, and the number of selected stars is appropriately counted, it 
would correct most of binarity's biases.


\paragraph{The hot Jupiter rate discrepancy}
There is at least one context in which ignoring binarity may already be 
leading to discrepant measurements.
Hot Jupiter occurrence rates measured by transit surveys ($\approx 0.5\%$) are 
marginally lower than those found by radial velocity surveys ($\approx 1\%$; 
see Table~\ref{tab:hj_rates}).
The discrepancy has weak statistical significance ($<3\sigma$).
One reason to expect a difference is that the corresponding stellar 
populations have distinct metallicities.
As argued by Gould et al. (2006), the RV sample is biased towards 
metal-rich stars, which have been measured by RV surveys to preferentially 
host more giant planets (Santos et al 2004, Fischer and Valenti 2005).
The {\it Kepler}\ sample specifically has been measured to be more metal poor 
than the local neighborhood, with a mean metallicity of $[{\rm M/H}]_{\rm 
    mean}\approx -0.05$ (Dong et al., 2014; Guo et al., 2017).
Studying the problem in detail, Guo et al. recently argued that the 
metallicity difference could account for a $\approx 0.1\%$ difference in the 
measured rates between the CKS and {\it Kepler}\ samples~--~not a $\approx 
0.5\%$ difference.
Guo et al. concluded that ``other factors, such as binary contamination and 
imperfect stellar properties'' must also be at play.

Aside from surveying stars of varying metallicities, radial velocity and 
transit surveys differ in how they treat binarity.
Radial velocity surveys typically reject both visual and spectroscopic binaries
({\it e.g.}, Wright et al. 2012).
Transit surveys typically observe binaries, but the question of whether they 
were searchable to begin with is left for later interpretation.
In spectroscopic follow-up of candidate transiting planets, the prevalence of 
astrophysical false-positives may also lead to a bias against confirmation of 
transiting planets in binary systems.

Ignoring these complications, in this work we showed that
binarity biases transit survey occurrence rates through its effects on 
completeness, star counts, and the apparent radii of detected planets.
Specifically, our results from Sec.~\ref{sec:model_3} indicate that binarity 
could lead to underestimated HJ rates by a multiplicative factor of $\approx 
1.3$.

To assess the effect this might have towards resolving the hot Jupiter rate 
discrepancy we ask:
what is the probability of Wright et al. (2012)'s result, given a rate drawn 
from the stated bounds of Petigura et al. (in prep)? (See 
Table~\ref{tab:hj_rates}).
In other words, we first take the true HJ rate per thousand stars as 
$\Lambda_{\rm HJ} = 5.7 \pm 1.3$, with Gaussian uncertainties. 
We then draw from a Poisson distribution and compute the probability of 
detecting at least 10 hot Jupiters in a sample of 836 stars.
Without accounting for binarity or metallicity, only 4\% of RV surveys would 
detect at least 10 hot Jupiters.
If we multiply $\Lambda_{\rm HJ}$ by $1.2$ to account for Guo et al. 
(2017)'s measured metallicity difference between the {\it Kepler}\ field and 
the local solar neighborhood, 9\% of RV surveys would detect at least 10 hot 
Jupiters.
If we multiply once more by $1.3$ to account for binarity's bias, we find that
23\% of RV surveys would detect at least 10 hot Jupiters, and any discrepancy 
would be rather tenuous.
We emphasize that this result is only suggestive~--~a true resolution of the 
rate discrepancy would likely require a detailed understanding of the {\it 
Kepler}\ field's multiplicity statistics.


\paragraph{The rate of Earth analogs}
Per {\it Kepler}'s primary science objective, the rate of Earth-like planets 
orbiting Sun-like stars has been independently measured by Youdin, Petigura, 
Dong \& Zhu, Foreman-Mackey et al., and Burke et al. (2011, 2013, 2013, 2014, 
and 2015 respectively).
These efforts have found that the one-year terrestrial planet occurrence rate 
varies between $\approx 0.03$ and $\approx 1$ per Sun-like star, depending on 
assumptions that are made when retrieving the rate (Burke et al. 2015's 
Fig.~17).
In our Model \#3, the inferred rate is $\approx 0.84\times$ the true rate 
around single stars.
This bias is quite small compared to the other systematic factors that 
currently dominate the dispersion in $\eta_\oplus$ measurements.
If a future transit survey measures an absolute value of $\eta_\oplus$ to 
better than a factor of two, binarity might merit closer attention.


\paragraph{The CKS radius gap}
By improving the stellar radii... {\bf TODO}
%TODO DO THESE CALCULATIONS


\paragraph{Does a detected planet orbit the primary or secondary?}
Ciardi et al. (2015) studied the effects of stellar multiplicity on the 
planet radii derived from transit surveys.
They modeled the problem for {\it Kepler}\ objects of interest by matching a 
population of binary and tertiary companions to KOI stars, 
under the assumption that the KIC-listed stars were the primaries.
They then computed planet radius correction factors assuming that {\it 
Kepler}-detected planets orbited the primary or companion stars
with equal probability (their Sec. 5).
Under these assumptions, they found that any given planet's radius is on 
average underestimated by a multiplicative factor of 1.5.

Our models show that assuming a detected planet has equal probability of 
orbiting the primary or secondary leads to an overstatement of
binarity's population-level effects.
A planet orbiting the secondary does lead to extreme corrections, but these 
cases are rare outliers, because the searchable volume for secondaries is so 
much smaller than that for primaries.
Phrased in terms of the completeness, in our Model \#3 only $\sim 6\%$ of 
selected secondaries are searchable, compared to $\sim 60\%$ of selected 
primaries.
This means that when high-resolution imaging discovers a binary companion in 
a system that hosts a detected transiting planet, the planet is much
more likely to orbit the primary.
This statement is independent of the fact that planets are often confirmed to 
orbit the primary by inferring the stellar density from the transit duration.


\paragraph{On the utility for future occurrence rate measurements}
{\it TESS}\ is expected to discover over $10^4$ giant planets (Sullivan 
et al. 2015).
Though they will be difficult to distinguish from false positives, one 
possible use of this overwhelmingly large sample will to measure an
occurrence rate of short-period giant planets.
Our work indicates that if this measurement is to be more accurate than $\sim 
30\%$, binarity cannot be neglected.


\paragraph{Independent approaches for estimating binarity's effects}
T. Barclay et al. (in prep) have performed the exercise of taking stars 
selected by the {\it Kepler}\ team, pairing them with a population of 
secondaries, injecting a realistic distribution of planet radii, 
and then comparing the inferred occurrence rates with the true ones.
In their model, they find that binarity leads to an inferred rate of 
Earth-sized planets $\approx 10\%$ less than the true rate.
In our Model \#3, if all $Z_i$'s are equal (a plausible assumption in 
the lack of evidence to the contrary), the underestimate is by a comparable 
16\%.